\documentclass[11pt]{article}

\usepackage{graphicx}
\usepackage{float}
\usepackage{geometry}
\geometry{margin=1in}

\title{StegDetector Automated Steganalysis Report}
\author{Mihai Bontea}
\date{\today}

\begin{document}
\maketitle

\section*{Input Image}
\textbf{Filename:} {{IMAGE_NAME}}

\section*{File Metadata Anomalies}

Sometimes the process of steganography leaves other, more noticeable traces besides the noise in the image itself. The image has been scanned for such traces:
structural, metadata, and container anomalies that might indicate tampering or hidden data.

Examples include unusual file size, invalid file headers, missing or strange EXIF metadata, and corrupted or suspicious PNG chunks. While these are not a direct proof of
steganography, they are heuristic red flags.

{{FILE_WARNINGS}}

\section*{RS Analysis}

\begin{figure}[H]
\centering
\includegraphics[width=0.9\linewidth]{rs_heatmap.pdf}
\caption{RS Analysis Heatmap}
\end{figure}

\begin{figure}[H]
\centering
\includegraphics[width=0.9\linewidth]{rs_overlay.pdf}
\caption{RS Overlay Visualization}
\end{figure}

\section*{High-Pass Residual Analysis}

\begin{figure}[H]
\centering
\includegraphics[width=0.95\linewidth]{high_pass_analysis.pdf}
\caption{High-Pass Residual Steganalysis}
\end{figure}

\section*{Detection Result}
\textbf{Estimated Stego Confidence:} {{CONFIDENCE}}

\end{document}
